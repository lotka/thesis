\lstdefinestyle{customc}{
  belowcaptionskip=1\baselineskip,
  breaklines=true,
  frame=L,
  xleftmargin=\parindent,
  language=C,
  showstringspaces=false,
  basicstyle=\footnotesize\ttfamily,
  keywordstyle=\bfseries\color{green!40!black},
  commentstyle=\itshape\color{purple!40!black},
  identifierstyle=\color{blue},
  stringstyle=\color{orange},
}
\newpage
\section{Joint Spectrum Code}
\begin{lstlisting}[style=customc, language=C]
#include <stdio.h>
#include <stdlib.h>
#include <string.h>
#include <time.h>
#include "TunicsInterface.h"
#include "OsaInterface.h"
#include "PiezoInterface.h"
#include "PowermeterInterface.h"
#include "Recoupling.h"
#include "VOAInterface.h"
#include "XTA50Interface.h"

// Turn off the laser and get a blank spectrum
void getBlankSpectrum(TunicsHandle laser, OsaHandle * osa, char * name, int amount);

// Prints out a 2D matrix representing the joint spectrum to file
void printJointSpectrumWavelengthData(char * filename, osaRawData ** spectrum, char delimter, int numberOfSeedReadings);

// Take spectral scan, main loop copied from the spectral scan program
int takeSpectralScan(float lmin, float lmax, float lstep, float attenuation,PowermeterHandle outmeter,PowermeterHandle outmeter2, TunicsHandle laser);

// Prints out a regular spectrum to file
void printSpectrum(char * filename, osaRawData * spectrum, char delimter);

void takeAndPrintJointSpectrum(OsaHandle * osa,
                               TunicsHandle laser,
                               PowermeterHandle tapPowerMeter,
                               PowermeterHandle chipPowerMeter,
                               float osaStartWavelength_nm,
                               float osaEndWavelength_nm,
                               float osaSampleNumber,
                               float startWavelength,
                               float endWavelength,
                               float numberOfMeasurements,
                               float seedResolution,
                               int manageCoupling,
                               PiezoHandle leftPiezo,
                               PiezoHandle rightPiezo,
                               float maximumZVoltage,
                               float currentAttenuationLin,
                               int JSAsToTake,
                               int jsaNumber,
                               int tunableFilterOnCW);

XTA50System XTA50;

/* Modes of opertation
        1.  Normal JSA
        2.      Tunable Filter on CW, Normal
        3.  Normal Attenuation Scan
        4.      Tunable Filter on CW, Attenuation Scan
        5.  Tunable Filter on Pump, FWHM Scan
*/

int main (int argc, char** argv)
{

	if (argc != 31)
	{
		fprintf (stderr, "Error: Incorrect number of inputs. Expected 22, received %i.\n", argc-1);
		return 1;
	}

    int manageCoupling;
    float startWavelength, endWavelength, seedResolution, maximumZVoltage;
    char laserCom[256];
    char osaConnectionDescripitor[256];
    char leftPiezoCom[256];
    char rightPiezoCom[256];
    char chipPowermeterCom[256];
    char tapPowermeterCom[256];
    float cwLaserPower_mW = 0;
    float pumpLaserCurrent_mA = 0;
    float EDFACurrent_mA = 0;
    int scanType = 0;
    float startAttenuation, endAttenuation, attenuationStep;
    int cwAWGChannel, pumpAWGChannel;
    char chipName[256];
    float pumpWavelength;
    float TunableFilterCWOffset;
    float FWHMStart;
    float FWHMStep;
    float FWHMEnd;
    char tunableFilterCom[256];
    char VOACom[256];
    float ChipPowermeterAttenuation;
    float TapPowermeterAttenuation;

    sscanf(argv[1], "%f", &startWavelength);
    sscanf(argv[2], "%f", &endWavelength);
    sscanf(argv[3], "%f", &seedResolution);
    sscanf(argv[4], "%i", &manageCoupling);
    sscanf(argv[5], "%f", &maximumZVoltage);
    strcpy(laserCom,argv[6]);
    strcpy(osaConnectionDescripitor,argv[7]);
    strcpy(leftPiezoCom,argv[8]);
    strcpy(rightPiezoCom,argv[9]);
    strcpy(chipPowermeterCom,argv[10]);
    strcpy(tapPowermeterCom,argv[11]);
    sscanf(argv[12], "%f", &cwLaserPower_mW);
    sscanf(argv[13], "%f", &pumpLaserCurrent_mA);
    sscanf(argv[14], "%f", &EDFACurrent_mA);
    sscanf(argv[15], "%d", &scanType);
    sscanf(argv[16], "%f", &startAttenuation);
    sscanf(argv[17], "%f", &endAttenuation);
    sscanf(argv[18], "%f", &attenuationStep);
    sscanf(argv[19], "%d", &cwAWGChannel);
    sscanf(argv[20], "%d", &pumpAWGChannel);
    strcpy(chipName,argv[21]);
    sscanf(argv[22], "%f", &pumpWavelength);
    sscanf(argv[23], "%f", &TunableFilterCWOffset);
    sscanf(argv[24], "%f", &FWHMStart);
    sscanf(argv[25], "%f", &FWHMStep);
    sscanf(argv[26], "%f", &FWHMEnd);
    strcpy(tunableFilterCom,argv[27]);
    strcpy(VOACom,argv[28]);
    sscanf(argv[29], "%f", &ChipPowermeterAttenuation);
    sscanf(argv[30], "%f", &TapPowermeterAttenuation);

    printf("min = %f\nmax= %f\nstep = %f\n", startWavelength,endWavelength,seedResolution);
    VOAHandle voaHandle = InitVOA(VOACom);

    // Set up devices
    OsaHandle * osa = InitOsa(osaConnectionDescripitor);

    //osaRawData * spec = osaGetSpectrum(osa);
    //printSpectrum("OSASpectrum950nm.txt", spec, ' ');
    //return 0;

    /* CW Laser */
    TunicsHandle laser = InitTunics(laserCom);
    PowermeterHandle chipPowerMeter = InitPowermeter(chipPowermeterCom);
    PowermeterHandle tapPowerMeter = InitPowermeter(tapPowermeterCom);

    /* Piezo for side coupling */
    PiezoHandle leftPiezo;
    PiezoHandle rightPiezo;
    if(manageCoupling == 1)
    {
        leftPiezo = InitPiezo(leftPiezoCom);
        rightPiezo = InitPiezo(rightPiezoCom);
    }


    /* Start the tunable filter */
    XTA50_SerialCom_type XTA50Vars;
    int useTunableFilterForCW = 0;
    if(scanType == 2 || scanType == 3 || scanType == 5)
    {
        sprintf(XTA50Vars.COMPort,tunableFilterCom);
        InitXTA50(XTA50Vars);
        if(scanType == 2 || scanType == 4)
        {
            useTunableFilterForCW = 1;
        }
    }

    // CW LasertapPowerMeter
    int numberOfMeasurements = (int)ceil((endWavelength-startWavelength)/seedResolution)+1;

    // Get important data form OSA
    float * tempArray;
    tempArray = getNumericalSettings(osa,"DCA?",3); // Get wavelength range and sample number
    float osaStartWavelength_nm = tempArray[0];
    float osaEndWavelength_nm = tempArray[1];
    float osaSampleNumber = (int)tempArray[2];

    tempArray = getNumericalSettings(osa,"RES?",1); // Get OSA resolution
    float osaResolution_nm = tempArray[0];

    char *  VBW = getTextSetting(osa,"VBW?"); // Get OSA video bandwidth
    char *  HDR = getTextSetting(osa,"DRG?"); // Get dynamic range mode

    // Create a big file with as much information about the joint spectrum as possible
    // Open files
    FILE * infoFile = fopen("info.txt", "w");
    fprintf(infoFile, "#General Information\n");
    fprintf(infoFile, "Chip : %s\n",chipName);
    // Get the time
    time_t current_time;
    char* c_time_string;
    current_time=time(NULL);
    c_time_string = ctime(&current_time);
    fprintf(infoFile, "Timestamp: ");
    fprintf(infoFile, c_time_string);
    fprintf(infoFile, "\n");
    fprintf(infoFile, "Notes : \n");
    fprintf(infoFile, "Recoupling enabled: %d\n\n", manageCoupling);
    fprintf(infoFile, "#CW Laser Parameters\n");
    fprintf(infoFile, "AWG Channel : %d\n",cwAWGChannel);
    fprintf(infoFile, "Start wavelength (nm): %f\n",startWavelength);
    fprintf(infoFile, "End wavelength (nm) : %f\n",endWavelength);
    fprintf(infoFile, "Wavelength step (nm) :  %f\n",seedResolution);
    fprintf(infoFile, "Sample number : %d\n",numberOfMeasurements);
    fprintf(infoFile, "Laser power (mW): %f\n\n",cwLaserPower_mW);

    fprintf(infoFile, "#Pump laser parameters\n");
    fprintf(infoFile, "AWG Channel : %d\n",pumpAWGChannel);
    fprintf(infoFile, "Wavelength (nm) : %f\n",pumpWavelength);
    fprintf(infoFile, "Current (mA): %f\n\n",pumpLaserCurrent_mA);

    fprintf(infoFile, "#EDFA parameters\n");
    fprintf(infoFile, "Current (mA): %d\n\n",EDFACurrent_mA);

    fprintf(infoFile, "# OSA Settings \n");
    fprintf(infoFile, "Start Wavelength (nm) : %f\n",osaStartWavelength_nm);
    fprintf(infoFile, "End Wavelength (nm) : %f\n",osaEndWavelength_nm);
    fprintf(infoFile, "Sample Number : %f\n",osaSampleNumber);
    fprintf(infoFile, "Resolution (nm) : %f\n",osaResolution_nm);
    fprintf(infoFile, "Video Bandwidth : %s",VBW);
    fprintf(infoFile, "Dynamic Range Mode: : %s\n",HDR);

    SetTunicsEmission (laser,1);
    if(useTunableFilterForCW)
    {
        SetLambda(startWavelength-0.08);
    }
    SetTunicsWavelength(laser,startWavelength);
    float initialChipPower = 10.0*log10(MeasurePowermeter(chipPowerMeter)) + 30 + ChipPowermeterAttenuation;
    float initialTapPower  = 10.0*log10(MeasurePowermeter(tapPowerMeter))  + 30 + TapPowermeterAttenuation;

    fprintf(infoFile,"#Powermeter initial readings\n");
    fprintf(infoFile, "Before Chip (dBm): %f\n",initialTapPower);
    fprintf(infoFile, "After Chip (dBm): %f\n",initialChipPower);
    fprintf(infoFile, "Loss (dBm): %f\n\n",initialTapPower-initialChipPower);

    system("MKDIR osaSpectrumsWithNoCW");
    system("MKDIR jointSpectrums");
    system("MKDIR powerLogs");
    //system("MKDIR spectralScans");

    /* Buffer for filenames */
    char str[500];
    // Normal JSA
    // Tunable Filter on CW, Normal
    if(scanType == 1 || scanType == 2)
    {
        getBlankSpectrum(laser,osa,"osaSpectrumsWithNoCW/noSeedSpectrumBefore",5);
        SetTunicsEmission(laser,1);
        //takeSpectralScan(startWavelength,endWavelength,seedResolution,0,chipPowerMeter,tapPowerMeter,laser);
        takeAndPrintJointSpectrum(osa,
                                  laser,
                                  tapPowerMeter,
                                  chipPowerMeter,
                                  osaStartWavelength_nm,
                                  osaEndWavelength_nm,
                                  osaSampleNumber,
                                  startWavelength,
                                  endWavelength,
                                  numberOfMeasurements,
                                  seedResolution,
                                  manageCoupling,
                                  leftPiezo,
                                  rightPiezo,
                                  maximumZVoltage,
                                  0,
                                  1,
                                  0,
                                  useTunableFilterForCW);
        getBlankSpectrum(laser,osa,"osaSpectrumsWithNoCW/noSeedSpectrumAfter",5);
        SetTunicsEmission(laser,1);
    }
    // Attenuation Scan
    // Tunable Filter on CW, Attenuation Scan
    if(scanType == 3 || scanType == 4)
    {
        fprintf(infoFile,"#Attenuator Settings\n");
        fprintf(infoFile, "Attenuation enabled?: %d\n",scanType);
        fprintf(infoFile, "Start Attenuation: %f\n",startAttenuation);
        fprintf(infoFile, "End Attenuation: %f\n",endAttenuation);
        fprintf(infoFile, "Attenuation Step: %f\n\n",attenuationStep);
        fprintf(infoFile,"#Attenuations\nno. watts DB Tap Power Chip Power\n");
        int numberOfJSAsToTake = (int)((endAttenuation - startAttenuation)/attenuationStep) + 1;
        float currentTapPower;
        float currentChipPower;
        SetDVA_attenuation(voaHandle,10*log10(startAttenuation));
        int i;
        for(i = 0; i < numberOfJSAsToTake; ++i)
        {
            sprintf(str,"osaSpectrumsWithNoCW/noSeedSpectrumBefore_%i_",i);

            getBlankSpectrum(laser,osa,str,5);
            SetTunicsEmission(laser,1);

            float currentAttenuationLin = startAttenuation + attenuationStep*(float)i;
            float currentAttenuation_dB = -10 * log10 (1-currentAttenuationLin);
            SetAttenuationLin(voaHandle,currentAttenuationLin);

            printf("Trying to set attenuator to %f dB\n", currentAttenuation_dB);
            currentTapPower = 10.0*log10(MeasurePowermeter(tapPowerMeter))  + 30 + TapPowermeterAttenuation;
            currentChipPower = 10.0*log10(MeasurePowermeter(chipPowerMeter))  + 30 + ChipPowermeterAttenuation;
            fprintf(infoFile, "%d %f %f %f %f\n",i,currentAttenuationLin,currentAttenuation_dB,currentTapPower,currentChipPower);
            //takeSpectralScan(startWavelength,endWavelength,seedResolution,currentAttenuation_dB,chipPowerMeter,tapPowerMeter,laser);
            takeAndPrintJointSpectrum(osa,
                                      laser,
                                      tapPowerMeter,
                                      chipPowerMeter,
                                      osaStartWavelength_nm,
                                      osaEndWavelength_nm,
                                      osaSampleNumber,
                                      startWavelength,
                                      endWavelength,
                                      numberOfMeasurements,
                                      seedResolution,
                                      manageCoupling,
                                      leftPiezo,
                                      rightPiezo,
                                      maximumZVoltage,
                                      currentAttenuationLin,
                                      numberOfJSAsToTake,
                                      i,
                                      useTunableFilterForCW);

            if(manageCoupling == 1)
            {
                if(GetPiezoVoltage(leftPiezo,'z') < maximumZVoltage)
                {
                    printf("Recoupling left piezo.\n");
                    RecoupleDynamic(chipPowerMeter,leftPiezo,100);
                }
                else
                {
                    printf("LEFT Z MAX REACHED!");
                }
                if(GetPiezoVoltage(rightPiezo,'z') < maximumZVoltage)
                {
                    printf("Recoupling right piezo.\n");
                    RecoupleDynamic(chipPowerMeter,rightPiezo,100);
                }
                else
                {
                    printf("RIGHT Z MAX REACHED!");
                }
            }
        }
        sprintf(str,"osaSpectrumsWithNoCW/noSeedSpectrumBefore_%i_",i);
        getBlankSpectrum(laser,osa,str,5);
        SetTunicsEmission(laser,1);
    }

    //Tunable Filter on Pump, FWHM Scan
    if(scanType == 5)
    {
        for(int i = 0; i < (FWHMEnd-FWHMStart)/FWHMStep + 1; ++i)
        {
            float FWHM = FWHMStart + (float)i*FWHMStep;
            SetFWHM(FWHM);
            printf("FWHM =  %f\n", FWHM);

            sprintf(str,"osaSpectrumsWithNoCW/noSeedSpectrumBefore_%i_",i);
            getBlankSpectrum(laser,osa,str,5);
            SetTunicsEmission(laser,1);

            fprintf(infoFile, "%d %f %f\n",i,FWHM,MeasurePowermeter(tapPowerMeter),MeasurePowermeter(chipPowerMeter));

            takeAndPrintJointSpectrum(osa,
                                      laser,
                                      tapPowerMeter,
                                      chipPowerMeter,
                                      osaStartWavelength_nm,
                                      osaEndWavelength_nm,
                                      osaSampleNumber,
                                      startWavelength,
                                      endWavelength,
                                      numberOfMeasurements,
                                      seedResolution,
                                      manageCoupling,
                                      leftPiezo,
                                      rightPiezo,
                                      maximumZVoltage,
                                      0,
                                      1,
                                      i,
                                      useTunableFilterForCW);
            sprintf(str,"osaSpectrumsWithNoCW/noSeedSpectrumAfter_%i_",i);
            getBlankSpectrum(laser,osa,str,5);
            SetTunicsEmission(laser,1);
        }
        if(manageCoupling == 1)
        {
            if(GetPiezoVoltage(leftPiezo,'z') < maximumZVoltage)
            {
                printf("Recoupling left piezo.\n");
                RecoupleDynamic(chipPowerMeter,leftPiezo,100);
            }
            else
            {
                printf("LEFT Z MAX REACHED!");
            }
            if(GetPiezoVoltage(rightPiezo,'z') < maximumZVoltage)
            {
                printf("Recoupling right piezo.\n");
                RecoupleDynamic(chipPowerMeter,rightPiezo,100);
            }
            else
            {
                printf("RIGHT Z MAX REACHED!");
            }
        }
        fclose(infoFile);
    }

    // Record chip transmission
    fprintf(infoFile,"#Powermeter final readings\n");
    if(useTunableFilterForCW)
    {
        SetLambda(startWavelength-0.08);
    }
    SetTunicsWavelength(laser,startWavelength);
    float chip = 10.0*log10(MeasurePowermeter(chipPowerMeter)) + 30 + ChipPowermeterAttenuation;
    float tap = 10.0*log10(MeasurePowermeter(tapPowerMeter)) + 30 + TapPowermeterAttenuation;
    fprintf(infoFile, "Before Chip (dBm): %f\n",tap);
    fprintf(infoFile, "After Chip (dBm): %f\n",chip);
    fprintf(infoFile, "Loss (dBm): %f\n\n",tap-chip);
    fclose(infoFile);


    //Disable the laser
    SetTunicsEmission(laser,0);

    if(manageCoupling == 1)
    {
        //Retract the piezos (Hopefully zero is backwards)
        RampToVoltage(leftPiezo,0.00,'z');
        RampToVoltage(rightPiezo,0.00,'z');
    }

    /* Close all the things */
    ClosePowermeter(tapPowerMeter);
    ClosePowermeter(chipPowerMeter);
    if(manageCoupling == 1)
    {
        ClosePiezo(leftPiezo,0);
        ClosePiezo(rightPiezo,0);
    }
    CloseTunics(laser);
    osaWrite(osa,"SRT"); // Tell the osa to keep taking spectrums
    //CloseOsa(osa);
    CloseXTA50();
    return 0;
}

void getBlankSpectrum(TunicsHandle laser, OsaHandle * osa, char * name, int amount)
{
    char str[500];
    SetTunicsEmission(laser,0);
    /* Get a spectrum after experiment, might come in handy */
    printf("Taking initial spectrum without seed laser.\n");
    osaRawData * spectrumWithoutSeedLaser;
    for(int i = 0; i < amount; ++i)
    {
        spectrumWithoutSeedLaser = osaGetSpectrum(osa);
        sprintf(str,"%s%d%s",name,i,".txt");
        printSpectrum(str,spectrumWithoutSeedLaser,',');
    }
}

void printSpectrum(char * filename, osaRawData * spectrum, char delimter)
{

    FILE * fp = fopen(filename, "w");
    float * singleSpectrum = parseRawDataIntoFloatArray(spectrum,spectrum->osaSampleNumber,'\n');
    for(int i = 0; i < spectrum->osaSampleNumber; ++i)
    {
        fprintf(fp,"%e %e\n", spectrum->osaStartWavelength + i*(spectrum->osaEndWavelength - spectrum->osaStartWavelength)/(float)spectrum->osaSampleNumber, singleSpectrum[i]);
    }

    fclose(fp);
}

void printJointSpectrumWavelengthData(char * filename, osaRawData ** spectrum, char delimter, int numberOfSeedReadings)
{
    /* Print out the wavelengths to files (Probably pointless) */
    #define BUFLEN 128

    char buffer[BUFLEN];
    strcpy(buffer,filename);
    strcpy(buffer,strtok(buffer,"."));

    char osaFilename[BUFLEN];
    strcpy(osaFilename,buffer);

    char seedFilename[BUFLEN];
    strcpy(seedFilename,buffer);

    strcat(osaFilename,"_osaWavelengths.txt");
    strcat(seedFilename,"_seedWavelengths.txt");

    FILE * osaWavelengths = fopen(osaFilename, "w");
    FILE * seedWavelengths = fopen(seedFilename, "w");

    for(int i = 0; i < spectrum[0]->osaSampleNumber; ++i)
    {
        float value = spectrum[0]->osaStartWavelength + (float)i*(spectrum[0]->osaEndWavelength - spectrum[0]->osaStartWavelength)/(float)(spectrum[0]->osaSampleNumber-1);
        fprintf(osaWavelengths, "%e\n", value);
    }

    for(int i = 0; i < spectrum[0]->numberOfSeedReadings; ++i)
    {
        float value = spectrum[i]->seedWavelength;
        fprintf(seedWavelengths, "%e\n", value);
    }

    fclose(osaWavelengths);
    fclose(seedWavelengths);
}

void takeAndPrintJointSpectrum(OsaHandle * osa,
                               TunicsHandle laser,
                               PowermeterHandle tapPowerMeter,
                               PowermeterHandle chipPowerMeter,
                               float osaStartWavelength_nm,
                               float osaEndWavelength_nm,
                               float osaSampleNumber,
                               float startWavelength,
                               float endWavelength,
                               float numberOfMeasurements,
                               float seedResolution,
                               int manageCoupling,
                               PiezoHandle leftPiezo,
                               PiezoHandle rightPiezo,
                               float maximumZVoltage,
                               float currentAttenuationLin,
                               int JSAsToTake,
                               int i,
                               int tunableFilterOnCW)
{

    float currentAttenuation_dB = -10 * log10 (1-currentAttenuationLin);


    // 2 Dimensional data
    osaRawData ** spectrum = NULL;

    // Alloc the data
    spectrum = malloc(sizeof(osaRawData)*numberOfMeasurements);
    char str[256];
    sprintf(str,"jointSpectrums/jsa_%d.txt",i);
    FILE * dataFile = fopen(str, "w");
    sprintf(str,"jointSpectrums/normalised_jsa_%d.txt",i);
    FILE * normalisedDataFile = fopen(str, "w");
    sprintf(str,"powerLogs/powerLog_%d.txt",i);
    FILE * powerFile = fopen(str, "w");

    fprintf(dataFile,"xrange %e %e\nyrange %e %e\ninvertrows\n", osaStartWavelength_nm, osaEndWavelength_nm, startWavelength, endWavelength);
    fprintf(normalisedDataFile,"xrange %e %e\nyrange %e %e\ninvertrows\n", osaStartWavelength_nm, osaEndWavelength_nm,startWavelength,endWavelength);

    float wavelength = startWavelength;

    float minsLeft = 0.0f;
    for(int i = 0; i < numberOfMeasurements; ++i)
    {
        int startTime = time(0);
        /* Set CW laser wavelength */
        if(tunableFilterOnCW)
        {
            SetLambda(wavelength-0.08);
        }
        SetTunicsWavelength(laser,wavelength);

        // Record the readings on the power meters
        float tapPower = MeasurePowermeter(tapPowerMeter)*100; //++20 accounts for the tapping
        float chipPower = MeasurePowermeter(chipPowerMeter)*(100.0/99.0);
        fprintf(powerFile,"%e,%e,%e\n",wavelength,tapPower,chipPower);
        printf("Current wavelength %.3f\n", wavelength);
        printf("Percentage: %.1f%\n", (double)100*i/(numberOfMeasurements-1));
        printf("Attenuation %.1f (dB)\n", currentAttenuation_dB);
        printf("Taking Spectrum...\n");

        spectrum[i] = osaGetSpectrum(osa);

        float normalisation = tapPower;
        if(normalisation < 0)
        {
            printf("Normalisation was negative, something is probably wrong with the set up!\n");
        }

        for(int j = 0; j < osaSampleNumber; ++j)
        {
            fprintf(dataFile,"%e", (spectrum[i]->floatData)[j]);
            fprintf(normalisedDataFile,"%e", (spectrum[i]->floatData)[j]/normalisation);
            if(j != osaSampleNumber - 1)
            {
                fprintf(dataFile, " ");
                fprintf(normalisedDataFile, " ");
            }
        }
        fprintf(dataFile,"\n");
        fprintf(normalisedDataFile,"\n");

        /* Transfer CW laser settings to the specturm struct */
        spectrum[i]->seedWavelength = wavelength;
        spectrum[i]->seedStartWavelength = startWavelength;
        spectrum[i]->seedEndWavelength = endWavelength;
        spectrum[i]->numberOfSeedReadings = (int)numberOfMeasurements;

        /* Update the wavelength for the next loop */
        wavelength += seedResolution;

        int endTime = time(0);
        minsLeft = (float)(endTime-startTime)*(float)(numberOfMeasurements-i+1+6)/(float)60;
        float totalTimeLeft = minsLeft*(float)(JSAsToTake-i);
        if(JSAsToTake > 1)
        {
            printf("This scan should finish in %.2f mins.\n The experiment should finish in %.2f mins.\n\n", minsLeft, totalTimeLeft);
        }
        else
        {
            printf("This scan should finish in %.2f mins.\n\n", minsLeft);;
        }
    }

    fclose(dataFile);
    fclose(normalisedDataFile);

    /* Data Formatting */
    printJointSpectrumWavelengthData("jsa.txt", spectrum,'\n',numberOfMeasurements);

    /* Freeing data and closing devices */
    for(int i = 0; i < numberOfMeasurements; ++i)
    {
        free(spectrum[i]->data);
        free(spectrum[i]->floatData);
        free(spectrum[i]);
    }

    free(spectrum);
}

int takeSpectralScan(float lmin, float lmax, float lstep, float attenuation,PowermeterHandle outmeter,PowermeterHandle outmeter2, TunicsHandle laser)
{
  char str[256];
  sprintf(str, "spectralScans/spectralScan%8.3f-%8.3f_att_%.3f.csv",lmin,lmax,attenuation);
  FILE * fp = fopen(str,"w");


  //Move to start wavelength and enable laser
  SetTunicsWavelength (laser, lmin);
  SetTunicsEmission (laser, 1);
  Sleep (100);


  //Scan
  printf ("Scanning from %8.3f to %8.3f in steps of %.3f.\n", lmin, lmax, lstep);
  float l = 0;
  double p1,p2;
  for (int i = 0; l < lmax; i++)
  {
    l = lmin + i*lstep;

    //SetTunicsWavelengthFast (laser, l, DWELL_TIME);
    SetTunicsWavelength (laser, l);

    p1 = MeasurePowermeter (outmeter); // Measure output
    p2 = MeasurePowermeter (outmeter2); // Measure output

    fprintf (fp, "%e, %e, %e\n", l, p1,p2);
  }

  fclose(fp);

  return 1;

}
\end{lstlisting}