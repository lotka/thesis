\newpage
\section{Results}
% General structure.
%   Display chip structure
% 	Display spectrums, try to calculate quality factor at least.
%   Display some prelimary purities.
\subsection{Glassgow}

This chip was used to do an initial proof of concept that the JSI of a ring resonator could be measured. Due to the fabrication process many of the spot size converters had varying levels of coupling, this imposed some restriction on the types of experiment which can be performed.
\begingroup
    \centering  
    \includegraphics[width=10cm]{img/results/glassgowChipNumbering.png}
    \captionof{figure}{Glassgow test structure chip}
     \vspace{3pt} \label{crossCompare}
\endgroup
% Display some joint spectrums. Estimate the power you were putting into the ring. 
\subsection{Toshiba}
\begingroup
    \centering  
    \includegraphics[width=10cm]{img/results/toshiba.png}
    \captionof{figure}{Glassgow test structure chip}
     \vspace{3pt} \label{crossCompare}
\endgroup
\subsubsection{Bistability Data}
% Should be easy
\subsubsection{Pulse shaping}
% Touch on the pulse shaping experiment
\subsubsection{Power Scans}
%
\subsection{a-Si}
\begingroup
    \centering  
    \includegraphics[width=10cm]{img/results/chipPictures/exampleASIRing.png}
    \captionof{figure}{Glassgow test structure chip}
     \vspace{3pt} \label{crossCompare}
\endgroup
% Some initial joint spectrums.
% Try to calculate the quality of the rings.
% Analyse the g2(0) data